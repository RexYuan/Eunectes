%%
%% This is file `sample-manuscript.tex',
%% generated with the docstrip utility.
%%
%% The original source files were:
%%
%% samples.dtx  (with options: `all,proceedings,bibtex,manuscript')
%%
%% IMPORTANT NOTICE:
%%
%% For the copyright see the source file.
%%
%% Any modified versions of this file must be renamed
%% with new filenames distinct from sample-manuscript.tex.
%%
%% For distribution of the original source see the terms
%% for copying and modification in the file samples.dtx.
%%
%% This generated file may be distributed as long as the
%% original source files, as listed above, are part of the
%% same distribution. (The sources need not necessarily be
%% in the same archive or directory.)
%%
%%
%% Commands for TeXCount
%TC:macro \cite [option:text,text]
%TC:macro \citep [option:text,text]
%TC:macro \citet [option:text,text]
%TC:envir table 0 1
%TC:envir table* 0 1
%TC:envir tabular [ignore] word
%TC:envir displaymath 0 word
%TC:envir math 0 word
%TC:envir comment 0 0
%%
%%
%% The first command in your LaTeX source must be the \documentclass
%% command.
%%
%% For submission and review of your manuscript please change the
%% command to \documentclass[manuscript, screen, review]{acmart}.
%%
%% When submitting camera ready or to TAPS, please change the command
%% to \documentclass[sigconf]{acmart} or whichever template is required
%% for your publication.
%%
%%
% \documentclass[manuscript,screen,review]{acmart}
\documentclass[acmsmall,sigconf]{acmart}

%%
%% \BibTeX command to typeset BibTeX logo in the docs
\AtBeginDocument{%
  \providecommand\BibTeX{{%
    Bib\TeX}}}

% %% Rights management information.  This information is sent to you
% %% when you complete the rights form.  These commands have SAMPLE
% %% values in them; it is your responsibility as an author to replace
% %% the commands and values with those provided to you when you
% %% complete the rights form.
% \setcopyright{acmlicensed}
% \copyrightyear{2018}
% \acmYear{2018}
% \acmDOI{XXXXXXX.XXXXXXX}

% %% These commands are for a PROCEEDINGS abstract or paper.
% \acmConference[Conference acronym 'XX]{Make sure to enter the correct
%   conference title from your rights confirmation emai}{June 03--05,
%   2018}{Woodstock, NY}
% %%
% %%  Uncomment \acmBooktitle if the title of the proceedings is different
% %%  from ``Proceedings of ...''!
% %%
% %%\acmBooktitle{Woodstock '18: ACM Symposium on Neural Gaze Detection,
% %%  June 03--05, 2018, Woodstock, NY}
% \acmISBN{978-1-4503-XXXX-X/18/06}


%%
%% Submission ID.
%% Use this when submitting an article to a sponsored event. You'll
%% receive a unique submission ID from the organizers
%% of the event, and this ID should be used as the parameter to this command.
%%\acmSubmissionID{123-A56-BU3}

%%
%% For managing citations, it is recommended to use bibliography
%% files in BibTeX format.
%%
%% You can then either use BibTeX with the ACM-Reference-Format style,
%% or BibLaTeX with the acmnumeric or acmauthoryear sytles, that include
%% support for advanced citation of software artefact from the
%% biblatex-software package, also separately available on CTAN.
%%
%% Look at the sample-*-biblatex.tex files for templates showcasing
%% the biblatex styles.
%%

%%
%% The majority of ACM publications use numbered citations and
%% references.  The command \citestyle{authoryear} switches to the
%% "author year" style.
%%
%% If you are preparing content for an event
%% sponsored by ACM SIGGRAPH, you must use the "author year" style of
%% citations and references.
%% Uncommenting
%% the next command will enable that style.
%%\citestyle{acmauthoryear}

\usepackage{mathtools}
\usepackage{multirow}

\graphicspath{{images/}}
\DeclareGraphicsExtensions{.pdf}

\DeclarePairedDelimiter{\set}{\{}{\}}
\DeclarePairedDelimiter{\tuple}{(}{)}
\DeclarePairedDelimiter{\abs}{\lvert}{\rvert}
\renewcommand{\implies}{\rightarrow}
\newcommand{\db}{\mathcal{D}}
\newcommand{\model}{\mathcal{M}}
\newcommand{\priv}{R}
\newcommand{\pos}{\hat{P}}
\newcommand{\posm}{\hat{P}_\mathcal{M}}
\newcommand{\tru}{T}
\newcommand{\leg}{L}
\newcommand{\sco}{\hat{S}}
\newcommand{\scom}{\hat{S}_\mathcal{M}}
\newcommand{\calib}{C}
\newcommand{\prob}[1]{\text{Pr}[#1]}

%%
%% end of the preamble, start of the body of the document source.
\begin{document}

%%
%% The "title" command has an optional parameter,
%% allowing the author to define a "short title" to be used in page headers.
\title{The Name of the Title Is Hope}

%%
%% The "author" command and its associated commands are used to define
%% the authors and their affiliations.
%% Of note is the shared affiliation of the first two authors, and the
%% "authornote" and "authornotemark" commands
%% used to denote shared contribution to the research.
\author{Chih-Cheng Rex Yuan}
\email{hello@rexyuan.com}
\affiliation{%
  \institution{Institute of Information Science, Academia Sinica}
  \city{Taipei}
  \country{Taiwan}
}

\author{Bow-Yaw Wang}
\email{bywang@iis.sinica.edu.tw}
\affiliation{%
  \institution{Institute of Information Science, Academia Sinica}
  \city{Taipei}
  \country{Taiwan}
}

%%
%% By default, the full list of authors will be used in the page
%% headers. Often, this list is too long, and will overlap
%% other information printed in the page headers. This command allows
%% the author to define a more concise list
%% of authors' names for this purpose.
% \renewcommand{\shortauthors}{Trovato et al.}

%%
%% The abstract is a short summary of the work to be presented in the
%% article.
\begin{abstract}
  abstract
\end{abstract}

% %%
% %% The code below is generated by the tool at http://dl.acm.org/ccs.cfm.
% %% Please copy and paste the code instead of the example below.
% %%
% \begin{CCSXML}
% <ccs2012>
%  <concept>
%   <concept_id>00000000.0000000.0000000</concept_id>
%   <concept_desc>Do Not Use This Code, Generate the Correct Terms for Your Paper</concept_desc>
%   <concept_significance>500</concept_significance>
%  </concept>
%  <concept>
%   <concept_id>00000000.00000000.00000000</concept_id>
%   <concept_desc>Do Not Use This Code, Generate the Correct Terms for Your Paper</concept_desc>
%   <concept_significance>300</concept_significance>
%  </concept>
%  <concept>
%   <concept_id>00000000.00000000.00000000</concept_id>
%   <concept_desc>Do Not Use This Code, Generate the Correct Terms for Your Paper</concept_desc>
%   <concept_significance>100</concept_significance>
%  </concept>
%  <concept>
%   <concept_id>00000000.00000000.00000000</concept_id>
%   <concept_desc>Do Not Use This Code, Generate the Correct Terms for Your Paper</concept_desc>
%   <concept_significance>100</concept_significance>
%  </concept>
% </ccs2012>
% \end{CCSXML}

% \ccsdesc[500]{Do Not Use This Code~Generate the Correct Terms for Your Paper}
% \ccsdesc[300]{Do Not Use This Code~Generate the Correct Terms for Your Paper}
% \ccsdesc{Do Not Use This Code~Generate the Correct Terms for Your Paper}
% \ccsdesc[100]{Do Not Use This Code~Generate the Correct Terms for Your Paper}

% %%
% %% Keywords. The author(s) should pick words that accurately describe
% %% the work being presented. Separate the keywords with commas.
% \keywords{Do, Not, Us, This, Code, Put, the, Correct, Terms, for,
%   Your, Paper}

\received{20 February 2007}
\received[revised]{12 March 2009}
\received[accepted]{5 June 2009}

%%
%% This command processes the author and affiliation and title
%% information and builds the first part of the formatted document.
\maketitle

\section{Introduction}

\section{Related Work}

\clearpage

\section{Auditing Framework}

Our framework considers a scenario with three parties-data provider, model maker, and 3rd party auditor. The data provider has access to real data; for example, a census bureau. The model maker have AI models; for example, an AI company. The 3rd party auditor takes the data from data provider and AI models from model makers and perform fairness audits on them; for example, an investigative journalist.

In our original framework\cite{yuan2024ensuring}, after obtaining real data from data provider, the 3rd party auditor holds onto the real data for performing fairness audits. However, this may introduce privacy concerns such as security breach of the auditor.

Thus, we introduce a new framework where the auditor generates synthetic data based on real data upon retrival of the real data, and then holds onto the synthetic data and discards the real data, preventing further privacy breaches.

\subsection{Preliminaries}

A \emph{row} $r_i$ is a lookup table or dictionary. A \emph{database} $\db = \set{r_1, r_2, ...}$ is a collection of rows. The attributes of $\db$ is $\mathcal{A} = \set{A_1, A_2, ...}$. The domain of $A_i$ is $\Omega_i$.

For fairness measures\cite{yuan2024ensuring,pessach2022review}, let $Y$ to denote the ground truth of an outcome, let $\hat{Y}$ to denote the predicated result of an outcome, let $S$ denote protected attribute, and let $\epsilon$ denote some threshold. For non-binary prediction, such as a score, we use $\hat{V}$.

Let $C \subseteq \mathcal{A}$. Let $\Omega_C = \Pi_{i \in C} \Omega_i$. The \emph{marginal}\cite{barak2007privacy,mckenna2021winning} on $C$ is a vector $\mu \in \mathbb{R}^{\abs{\Omega_C}}$, indexed by domain element $t \in \Omega_C$, such that each entry is a count $\mu_t = \Sigma_{x \in \db} \vmathbb{1} [x_C = t]$ where $\vmathbb{1}$ is the indicator function. Let $M_C(\db)$ be the function that computes the marginal on $C$, i.e., $\mu = M_C(\db)$.

A \emph{randomized mechanism} is a randomized algorithm $M$ that takes a database $\db$ and, after, introducing noise, outputs some results in set $R$.

The $p$-norm is denoted by $L_p$ and the $p$-norm of a vector $x$ is denoted by $\Vert x \Vert_p$.

The normal distribution or Gaussian distribution with mean $\mu$ and standard deviation $\sigma$ is denoted by $\mathcal{N}(\mu, \sigma^2)$.

The Kullback–Leibler divergence between probability distributions $P$ and $Q$ is denoted by $D_{KL}(P \Vert Q)$. The generalization of it, R\'enyi divergence\cite{van2014renyi}, of order $\alpha$ is denoted by $D_\alpha(P \Vert Q)$.

\subsection{Fairness Measures}

We consider in this work various fairness measures listed in Table~\ref{tab:measures}. They can be broadly categorized into independence, separation, and sufficiency.

\begin{definition}[Independence\cite{barocas2023fairness}]\label{def:independence}
$(S, \hat{Y})$ satisfy independence if and only if $S \bot \hat{Y}$; that is
\[
P[\hat{Y} = 1 | S = 1] = P[\hat{Y} = 1 | S \neq 1]
\]
A relaxation of independence on a threshold is
\[
\abs{P[\hat{Y} = 1 | S = 1] - P[\hat{Y} = 1 | S \neq 1]} \leq \epsilon
\]
\end{definition}

\begin{definition}[Separation\cite{barocas2023fairness}]\label{def:separation}
$(S, Y, \hat{Y})$ satisfy separation if and only if $S \bot \hat{Y} | Y$; that is
\begin{align*}
P[\hat{Y} = 1 | S = 1, Y = 1] & = P[\hat{Y} = 1 | S \neq 1, Y = 1] \\
P[\hat{Y} = 1 | S = 1, Y = 0] & = P[\hat{Y} = 1 | S \neq 1, Y = 0]
\end{align*}
A relaxation of independence on a threshold is
\begin{align*}
\abs{P[\hat{Y} = 1 | S = 1, Y = 1] - P[\hat{Y} = 1 | S \neq 1, Y = 1]} & \leq \epsilon \\
\abs{P[\hat{Y} = 1 | S = 1, Y = 0] - P[\hat{Y} = 1 | S \neq 1, Y = 0]} & \leq \epsilon
\end{align*}
\end{definition}

\begin{definition}[Sufficiency\cite{barocas2023fairness}]\label{def:separation}
$(S, Y, \hat{Y})$ satisfy sufficiency if and only if $S \bot Y | \hat{Y}$; that is
\begin{align*}
P[Y = 1 | S = 1, \hat{Y} = 1] & = P[Y = 1 | S \neq 1, \hat{Y} = 1] \\
P[Y = 1 | S = 1, \hat{Y} = 0] & = P[Y = 1 | S \neq 1, \hat{Y} = 0]
\end{align*}
A relaxation of independence on a threshold is
\begin{align*}
\abs{P[Y = 1 | S = 1, \hat{Y} = 1] - P[Y = 1 | S \neq 1, \hat{Y} = 1]} & \leq \epsilon \\
\abs{P[Y = 1 | S = 1, \hat{Y} = 0] - P[Y = 1 | S \neq 1, \hat{Y} = 0]} & \leq \epsilon
\end{align*}
\end{definition}

\begin{table*}[h]
\caption{Fairness measures.}
\label{tab:measures}
\begin{tabular}{llc} % 'l' specifies left alignment for the first column
\toprule
\textbf{Category} & \textbf{Fairness Measure} & \textbf{Definition} \\
\midrule
\multirow{4}{*}{Independence} & Disparate Impact & $\frac{P[\hat{Y} = 1 | S \neq 1]}{P[\hat{Y} = 1 | S = 1]} \geq 1 - \epsilon$ \\
& Demographic Parity & $\abs{P[\hat{Y} = 1 | S = 1] - P[\hat{Y} = 1 | S \neq 1]} \leq \epsilon$ \\
& Conditional Statistical Parity & $\abs{P[\hat{Y} = 1 | S = 1, L = l] - P[\hat{Y} = 1 | S \neq 1, L = l]} \leq \epsilon$ \\
& Mean Difference & $\abs{E[\hat{Y}|S = 1] - E[\hat{Y}|S \neq 1]} \leq \epsilon$ \\
\multirow{4}{*}{Separation} & \multirow{2}{*}{Equalized Odds} & $\abs{P[\hat{Y} = 1 | S = 1, Y = 0] - P[\hat{Y} = 1 | S \neq 1, Y = 0]} \leq \epsilon$ \\
& & $\abs{P[\hat{Y} = 1 | S = 1, Y = 1] - P[\hat{Y} = 1 | S \neq 1, Y = 1]} \leq \epsilon$ \\
& Equal Opportunity & $\abs{P[\hat{Y} = 1 | S = 1, Y = 1] - P[\hat{Y} = 1 | S \neq 1, Y = 1]} \leq \epsilon$ \\
& Predictive Equality & $\abs{P[\hat{Y} = 1 | S = 1, Y = 0] - P[\hat{Y} = 1 | S \neq 1, Y = 0]} \leq \epsilon$ \\
\multirow{4}{*}{Sufficiency} & \multirow{2}{*}{Conditional Use Accuracy Equality} & $\abs{P[Y = 1 | S = 1, \hat{Y} = 1] - P[Y = 1 | S \neq 1, \hat{Y} = 1]} \leq \epsilon$ \\
& & $\abs{P[Y = 0 | S = 1, \hat{Y} = 0] - P[Y = 0 | S \neq 1, \hat{Y} = 0]} \leq \epsilon$ \\
& Predictive Parity & $\abs{P[Y = 1 | S = 1, \hat{Y} = 1] - P[Y = 1 | S \neq 1, \hat{Y} = 1]} \leq \epsilon$ \\
& Equal Calibration & $\abs{P[Y = 1 | S = 1, \hat{V} = v] - P[Y = 1 | S \neq 1, \hat{V} = v]} \leq \epsilon$ \\
\multirow{3}{*}{N/A} & Overall Accuracy Equality & $\abs{P[Y = \hat{Y} | S = 1] - P[Y = \hat{Y} | S \neq 1]} \leq \epsilon$ \\
& Positive Balance & $\abs{E[\hat{V} | Y = 1, S = 1] - E[\hat{V} | Y = 1, S \neq 1]} \leq \epsilon$ \\
& Negative Balance & $\abs{E[\hat{V} | Y = 0, S = 1] - E[\hat{V} | Y = 0, S \neq 1]} \leq \epsilon$ \\
\bottomrule
\end{tabular}
\end{table*}

\subsection{Differential Privacy}

\begin{definition}[Sensitivity\cite{dwork2014algorithmic}]\label{def:sensitivity}
Let $f$ be a function that takes a database $\db$ and outputs a vector $\mathbb{R}^p$. The $L_2$ sensitivity of $f$ is for all databases $\db_1,\db_2$ that differ in exactly one row:
\[
\Delta^2_f = max_{\db_1,\db_2} \Vert f(\db_1) - f(\db_2) \Vert_p
\]
\end{definition}

\begin{definition}[Gaussian Mechanism\cite{dwork2014algorithmic}]\label{def:gm}
Let $f$ be a function that takes a database $\db$ and outputs a vector $\mathbb{R}^p$. The Gaussian Mechanism $M$ adds Gaussian noise with scale $\sigma$ to each of the $p$ outputs:
\[
M(\db) = f(\db) + \mathcal{N}(0, \sigma^2 \mathbb{I})
\]
\end{definition}

\begin{definition}[Differential Privacy (DP) \cite{dwork2006calibrating,dwork2014algorithmic,mckenna2021winning}]\label{def:rdp}
A randomized mechanism $M$ satisfies $(\epsilon,\delta)$-DP if, for all databases $\db_1,\db_2$ that differ in exactly one row and for all subsets $S$ of $R$, we have
\[
\prob{M(\db_1) \in S} \leq e^\epsilon \prob{M(\db_2) \in S} + \delta
\]
\end{definition}

\begin{definition}[R\'enyi Differential Privacy (RDP)]\label{def:rdp}
A randomized mechanism $M$ satisfies $(\alpha,\gamma)$-RDP for $\alpha \geq 1$ and $\gamma \geq 1$ if, for all databases $\db_1,\db_2$ that differ in exactly one row, we have
\[
D_\alpha(M(\db_1) \Vert M(\db_2)) \leq \gamma
\]
\end{definition}

\begin{theorem}[RDP of the Gaussian Mechanism\cite{feldman2018privacy,mironov2017renyi}]
The Gaussian Mechanism satisfies $(\alpha, \alpha \frac{\Delta^2_f}{2 \sigma^2})$-RDP.
\end{theorem}

\section{Methodology}

\subsection{Differentially Private Synthetic Data}

\subsection{Fairness Checker}

\subsection{Implementation}

\section{Results}

\subsection{Adult Income Dataset}

\subsection{COMPAS Dataset}

\subsection{One More Dataset}

\section{Discussion}

\subsection{Accuracy}

\subsection{Impossibility}

\section{Conclusion}

Some examples.  A paginated journal article \cite{Abril07}

% \begin{table}
%   \caption{Frequency of Special Characters}
%   \label{tab:freq}
%   \begin{tabular}{ccl}
%     \toprule
%     Non-English or Math&Frequency&Comments\\
%     \midrule
%     \O & 1 in 1,000& For Swedish names\\
%     $\pi$ & 1 in 5& Common in math\\
%     \$ & 4 in 5 & Used in business\\
%     $\Psi^2_1$ & 1 in 40,000& Unexplained usage\\
%   \bottomrule
% \end{tabular}
% \end{table}

% \begin{table*}
%   \caption{Some Typical Commands}
%   \label{tab:commands}
%   \begin{tabular}{ccl}
%     \toprule
%     Command &A Number & Comments\\
%     \midrule
%     \texttt{{\char'134}author} & 100& Author \\
%     \texttt{{\char'134}table}& 300 & For tables\\
%     \texttt{{\char'134}table*}& 400& For wider tables\\
%     \bottomrule
%   \end{tabular}
% \end{table*}

\begin{figure}[h]
  \centering
  \includegraphics[width=\linewidth]{compas_mst}
  \caption{1907 Franklin Model D roadster. Photograph by Harris \&
    Ewing, Inc. [Public domain], via Wikimedia
    Commons. (\url{https://goo.gl/VLCRBB}).}
  \Description{A woman and a girl in white dresses sit in an open car.}
\end{figure}

\begin{figure}[h]
  \centering
  \includegraphics[width=\linewidth]{adult_mst}
  \caption{1907 Franklin Model D roadster. Photograph by Harris \&
    Ewing, Inc. [Public domain], via Wikimedia
    Commons. (\url{https://goo.gl/VLCRBB}).}
  \Description{A woman and a girl in white dresses sit in an open car.}
\end{figure}

%%
%% The acknowledgments section is defined using the "acks" environment
%% (and NOT an unnumbered section). This ensures the proper
%% identification of the section in the article metadata, and the
%% consistent spelling of the heading.
% \begin{acks}
% To Robert, for the bagels and explaining CMYK and color spaces.
% \end{acks}

%%
%% The next two lines define the bibliography style to be used, and
%% the bibliography file.
\bibliographystyle{ACM-Reference-Format}
\bibliography{references}


%%
%% If your work has an appendix, this is the place to put it.
% \appendix

\end{document}
\endinput
%%
%% End of file `sample-manuscript.tex'.
